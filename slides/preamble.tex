% xcolor and define colors -------------------------
\usepackage[table]{xcolor}

% https://www.viget.com/articles/color-contrast/
\definecolor{purple}{HTML}{5601A4}
\definecolor{navy}{HTML}{0D3D56}
\definecolor{ruby}{HTML}{9a2515}
\definecolor{alice}{HTML}{107895}
\definecolor{daisy}{HTML}{EBC944}
\definecolor{coral}{HTML}{F26D21}
\definecolor{kelly}{HTML}{829356}
\definecolor{cranberry}{HTML}{E64173}
\definecolor{jet}{HTML}{131516}
\definecolor{asher}{HTML}{555F61}
\definecolor{slate}{HTML}{314F4F}

% Mixtape Sessions
\definecolor{picton-blue}{HTML}{00b7ff}
\definecolor{violet-red}{HTML}{ff3881}
\definecolor{sun}{HTML}{ffaf18}
\definecolor{electric-violet}{HTML}{871EFF}

% Main theme colors
\definecolor{accent}{HTML}{00b7ff}
\definecolor{accent2}{HTML}{871EFF}
\definecolor{gray100}{HTML}{f3f4f6}
\definecolor{gray800}{HTML}{1F292D}

\definecolor{bgRaspberry}{HTML}{ff96a9}
\definecolor{bgCranberry}{HTML}{fda4d0}
\definecolor{bgOrange}{HTML}{f6b97b}
\definecolor{bgPurple}{HTML}{adb4f4}
\definecolor{bgBlue}{HTML}{cdfbff}
\definecolor{bgGreen}{HTML}{8ee7af}
\definecolor{bgRose}{HTML}{fecdd4}
\definecolor{bgYellow}{HTML}{ffea88}

% Beamer Options -------------------------------------

% Background
\setbeamercolor{background canvas}{bg = white}

% Change text margins
\setbeamersize{text margin left = 15pt, text margin right = 15pt}

% \alert
\setbeamercolor{alerted text}{fg = accent2}

% Frame title
\setbeamercolor{frametitle}{bg = white, fg = jet}
\setbeamercolor{framesubtitle}{bg = white, fg = accent}
\setbeamerfont{framesubtitle}{size = \small, shape = \itshape}

% Block
\setbeamercolor{block title}{fg = white, bg = accent2}
\setbeamercolor{block body}{fg = gray800, bg = gray100}

% Title page
\setbeamercolor{title}{fg = gray800}
\setbeamercolor{subtitle}{fg = accent}

%% Custom \maketitle and \titlepage
\setbeamertemplate{title page}
{
    %\begin{centering}
        \vspace{20mm}
        {\Large \usebeamerfont{title}\usebeamercolor[fg]{title}\inserttitle}\\
        {\large \itshape \usebeamerfont{subtitle}\usebeamercolor[fg]{subtitle}\insertsubtitle}\\ \vspace{10mm}
        {\insertauthor}\\
        {\color{asher}\small{\insertdate}}\\
    %\end{centering}
}

% Table of Contents
\setbeamercolor{section in toc}{fg = accent!70!jet}
\setbeamercolor{subsection in toc}{fg = jet}

% Button
\setbeamercolor{button}{bg = accent}

% Remove navigation symbols
\setbeamertemplate{navigation symbols}{}

% Table and Figure captions
\setbeamercolor{caption}{fg=jet!70!white}
\setbeamercolor{caption name}{fg=jet}
\setbeamerfont{caption name}{shape = \itshape}

% Bullet points

%% Fix left-margins
\settowidth{\leftmargini}{\usebeamertemplate{itemize item}}
\addtolength{\leftmargini}{\labelsep}

%% enumerate item color
\setbeamercolor{enumerate item}{fg = accent}
\setbeamerfont{enumerate item}{size = \small}
\setbeamertemplate{enumerate item}{\insertenumlabel.}

%% itemize
\setbeamercolor{itemize item}{fg = accent!70!white}
\setbeamerfont{itemize item}{size = \small}
\setbeamertemplate{itemize item}[circle]

%% right arrow for subitems
\setbeamercolor{itemize subitem}{fg = accent!60!white}
\setbeamerfont{itemize subitem}{size = \small}
\setbeamertemplate{itemize subitem}{$\rightarrow$}

\setbeamertemplate{itemize subsubitem}[square]
\setbeamercolor{itemize subsubitem}{fg = jet}
\setbeamerfont{itemize subsubitem}{size = \small}


% Special characters

\usepackage{collectbox}

\makeatletter
\newcommand{\mybox}{%
    \collectbox{%
        \setlength{\fboxsep}{1pt}%
        \fbox{\BOXCONTENT}%
    }%
}
\makeatother





% Links ----------------------------------------------

\usepackage{hyperref}
\hypersetup{
  colorlinks = true,
  linkcolor = accent2,
  filecolor = accent2,
  urlcolor = accent2,
  citecolor = accent2,
}


% Line spacing --------------------------------------
\usepackage{setspace}
\setstretch{1.1}


% \begin{columns} -----------------------------------
\usepackage{multicol}


% Fonts ---------------------------------------------
% Beamer Option to use custom fonts
\usefonttheme{professionalfonts}

% \usepackage[utopia, smallerops, varg]{newtxmath}
% \usepackage{utopia}
\usepackage[sfdefault,light]{roboto}

% Small adjustments to text kerning
\usepackage{microtype}



% Remove annoying over-full box warnings -----------
\vfuzz2pt
\hfuzz2pt


% Table of Contents with Sections
\setbeamerfont{myTOC}{series=\bfseries, size=\Large}
\AtBeginSection[]{
        \frame{
            \frametitle{Roadmap}
            \tableofcontents[current]
        }
    }


% Tables -------------------------------------------
% Tables too big
% \begin{adjustbox}{width = 1.2\textwidth, center}
\usepackage{adjustbox}
\usepackage{array}
\usepackage{threeparttable, booktabs, adjustbox}

% Fix \input with tables
% \input fails when \\ is at end of external .tex file
\makeatletter
\let\input\@@input
\makeatother

% Tables too narrow
% \begin{tabularx}{\linewidth}{cols}
% col-types: X - center, L - left, R -right
% Relative scale: >{\hsize=.8\hsize}X/L/R
\usepackage{tabularx}
\newcolumntype{L}{>{\raggedright\arraybackslash}X}
\newcolumntype{R}{>{\raggedleft\arraybackslash}X}
\newcolumntype{C}{>{\centering\arraybackslash}X}

% Figures

% \imageframe{img_name} -----------------------------
% from https://github.com/mattjetwell/cousteau
\newcommand{\imageframe}[1]{%
    \begin{frame}[plain]
        \begin{tikzpicture}[remember picture, overlay]
            \node[at = (current page.center), xshift = 0cm] (cover) {%
                \includegraphics[keepaspectratio, width=\paperwidth, height=\paperheight]{#1}
            };
        \end{tikzpicture}
    \end{frame}%
}

% subfigures
\usepackage{subfigure}


% Highlight slide -----------------------------------
% \begin{transitionframe} Text \end{transitionframe}
% from paulgp's beamer tips
\newenvironment{transitionframe}{
    \setbeamercolor{background canvas}{bg=accent!40!black}
    \begin{frame}\color{accent!10!white}\LARGE\centering
}{
    \end{frame}
}


% Table Highlighting --------------------------------
% Create top-left and bottom-right markets in tabular cells with a unique matching id and these commands will outline those cells
\usepackage[beamer,customcolors]{hf-tikz}
\usetikzlibrary{calc, fit, shadows, arrows, arrows.meta, shapes.misc, shapes,decorations, decorations.pathreplacing, positioning}
\usepackage[most,skins]{tcolorbox}
\tcbuselibrary{breakable}
\tcbset{
  highlight math/.style={
    notitle, enhanced,
    on line, boxsep=2pt, left=0pt, right=0pt, top=0pt, bottom=0pt,
    colback = bgRaspberry, colframe = white,
  }
}

% To set the hypothesis highlighting boxes red.
\newcommand\marktopleft[1]{%
    \tikz[overlay,remember picture]
        \node (marker-#1-a) at (0,1.5ex) {};%
}
\newcommand\markbottomright[1]{%
    \tikz[overlay,remember picture]
        \node (marker-#1-b) at (0,0) {};%
    \tikz[accent!80!jet, ultra thick, overlay, remember picture, inner sep=4pt]
        \node[draw, rectangle, fit=(marker-#1-a.center) (marker-#1-b.center)] {};%
}


% Define custom slide coordinate system ----------------------------------------
% https://tex.stackexchange.com/questions/89588/positioning-relative-to-page-in-tikz
% Defining a new coordinate system for the page:
%
% ┌──────────────────────────────────────────────┐
% │ (0, 0)                                (1, 0) │
% │                                              │
% │                                              │
% │                                              │
% │                                              │
% │                                              │
% │ (0, 1)                                (1, 1) │
% └──────────────────────────────────────────────┘
%
% source: https://tex.stackexchange.com/questions/89588/positioning-relative-to-page-in-tikz
%
\makeatletter
\def\parsecomma#1,#2\endparsecomma{\def\page@x{#1}\def\page@y{#2}}
\tikzdeclarecoordinatesystem{page}{
    \parsecomma#1\endparsecomma
    \pgfpointanchor{current page}{north west}
    % Save the upper left corner
    \pgf@xa=\pgf@x%
    \pgf@ya=\pgf@y%
    % save the lower right corner
    \pgfpointanchor{current page}{south east}
    \pgf@xb=\pgf@x%
    \pgf@yb=\pgf@y%
    % Transform to the correct placement
    \pgfmathparse{(\pgf@xb-\pgf@xa)*\page@x+\pgf@xa}
    \expandafter\pgf@x\expandafter=\pgfmathresult pt
    \pgfmathparse{(\pgf@yb-\pgf@ya)*\page@y+\pgf@ya}
    \expandafter\pgf@y\expandafter=\pgfmathresult pt
}
\makeatother

% To overlay on beamer slides, use the following:
%
% \begin{tikzpicture}[remember picture, overlay]
%   \node[anchor = north west] (anchor_name) at (page cs:0.0, 0.0) {
%     CONTENT HERE
%   };
% \end{tikzpicture}
%
% Note page cs is the coordinate system from above

% To help with placement, I will use `\devgrid` on a slide to figure out the coordinates of `page cs` to the 0.1
\newcommand{\devgrid}{
  \begin{tikzpicture}[remember picture, overlay]
    % vertical lines
    \draw (page cs: 0.1, 0.0) edge[black!20!white, line width = 0.3mm, dotted] (page cs: 0.1, 1.0);
    \draw (page cs: 0.2, 0.0) edge[black!20!white, line width = 0.3mm, dotted] (page cs: 0.2, 1.0);
    \draw (page cs: 0.3, 0.0) edge[black!20!white, line width = 0.3mm, dotted] (page cs: 0.3, 1.0);
    \draw (page cs: 0.4, 0.0) edge[black!20!white, line width = 0.3mm, dotted] (page cs: 0.4, 1.0);
    \draw (page cs: 0.5, 0.0) edge[black!20!white, line width = 0.3mm, dotted] (page cs: 0.5, 1.0);
    \draw (page cs: 0.6, 0.0) edge[black!20!white, line width = 0.3mm, dotted] (page cs: 0.6, 1.0);
    \draw (page cs: 0.7, 0.0) edge[black!20!white, line width = 0.3mm, dotted] (page cs: 0.7, 1.0);
    \draw (page cs: 0.8, 0.0) edge[black!20!white, line width = 0.3mm, dotted] (page cs: 0.8, 1.0);
    \draw (page cs: 0.9, 0.0) edge[black!20!white, line width = 0.3mm, dotted] (page cs: 0.9, 1.0);

    % horizontal lines
    \draw (page cs: 0.0, 0.1) edge[black!20!white, line width = 0.3mm, dotted] (page cs: 1.0, 0.1);
    \draw (page cs: 0.0, 0.2) edge[black!20!white, line width = 0.3mm, dotted] (page cs: 1.0, 0.2);
    \draw (page cs: 0.0, 0.3) edge[black!20!white, line width = 0.3mm, dotted] (page cs: 1.0, 0.3);
    \draw (page cs: 0.0, 0.4) edge[black!20!white, line width = 0.3mm, dotted] (page cs: 1.0, 0.4);
    \draw (page cs: 0.0, 0.5) edge[black!20!white, line width = 0.3mm, dotted] (page cs: 1.0, 0.5);
    \draw (page cs: 0.0, 0.6) edge[black!20!white, line width = 0.3mm, dotted] (page cs: 1.0, 0.6);
    \draw (page cs: 0.0, 0.7) edge[black!20!white, line width = 0.3mm, dotted] (page cs: 1.0, 0.7);
    \draw (page cs: 0.0, 0.8) edge[black!20!white, line width = 0.3mm, dotted] (page cs: 1.0, 0.8);
    \draw (page cs: 0.0, 0.9) edge[black!20!white, line width = 0.3mm, dotted] (page cs: 1.0, 0.9);
  \end{tikzpicture}
}
